\subheading{Wherein a Function To Fake the\\ Model Data is Written and Tested}
\section*{Section One and Two}

Before developing a function to identify the system a test function was
developed to later use on verifying the identification function.  The first step
in the development of this test function was creating a Matlab function that
takes in the $C$ and $\beta$ parameters that define the system along with a
frequency to evaluate it at ($\omega$) and a time to evaluate it for
($t_\text{end}$).  The function was simply Equation \eqref{y_model} evaulated at
the specified $C$, $\beta$ and $\omega$ with $t$ ranging from
$0$--$t_\text{end}$ at a frequency of $1$ kHz.

As a verification of this function an additional function utilising Matlab's
in-built \texttt{lsim} function was produced.  This function was based around
the transfer function from Equation \eqref{transfer-function} along with the input
from Equation \eqref{proportional-controller} to calculate the output.

Both models created were then plotted together with input values $C = 5$, $\beta
= 1$, $K_p = 60$, $f = 1.8$ and $t_\text{end} = 3$.  This can be seen in Fig.
\ref{section-two}.  Looking at this it can be seen that the steady state response of
the two systems is tending to be the same.  The major difference is in the
initial response, this is because the model used in the first equation is purely
a steady state model whereas the transfer function based model does base its
output off the initial conditions which are assumed to be zero.

\includefigure{width=0.6\textwidth}
              {images/section-two}
              {Model vs Matlabs simulated values\label{section-two}}
